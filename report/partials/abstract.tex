\begin{center}
	\textbf{Abstract}
\end{center}

This project aims to study the foundations of nonlinear dimensionality reduction through manifold learning with the algorithm known as Isometric Feature Mapping (ISOMAP) and observe the application of the algorithm in practical experiments. First, the relevant background will be presented in order to contextualize the reader and provide concepts that are closely related to dimensionality reduction, manifold learning and, of course, the ISOMAP algorithm. We will then proceed to study linear dimensionality reduction: algorithms, applications and limitations. Finally, manifold learning with ISOMAP will be covered, from the algorithm's concept to its extensions and limitations. In chapters 4 and 5, experiments created during the project are presented for both observation and comparison between the methods. The experiments were developed and executed in the following computational environment: linux Ubuntu 15.10 64 bits, Intel Core i7-4700MQ CPU 2.40GHz $\times$ 8, 16 GB of RAM. All the artifacts, (e.g., source-code, docs, experiments) can be found in the repository \url{https://github.com/lucasdavid/manifold-learning} and are licensed under the MIT License.

\afterpage{\blankpage}
